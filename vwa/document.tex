% Vorlage für die VWA. Version 20180117 (c) Leonard Michlmayr

% TODO: Wähle den für deine Arbeit die passenden Optionen!
\documentclass[DLS,
	inreferencehack,
	ohneVgl=false,
	ohneS=false,
	scauthor,
	rundeauslassung=false,
	bookstyle=false,
	widowlines=3]{vwa}

%% Trick texlipse to use biber instead of bibtex
\iffalse
\usepackage[error,backend=biber]{biblatex}
\fi
%%

% TODO: lade Zusatzpakete
\usepackage{textcomp}
\usepackage[output-decimal-marker={,}]{siunitx}

% TODO: Eigene Quellendatenbank laden.
\addbibresource{quellen.bib}

% TODO: Lege das Verzeichnis fest, wo Bilder liegen sollen
\graphicspath{{img/}}

% TODO: Eigenen Namen und Geschlecht wählen.
\Autor{Lorenz Brandstätter}
% TODO: Klasse einsetzen
\Klasse{8C}
% TODO: Betreuungslehrer oder Betreuungslehrerin einsetzen:
\Betreuer{OStR. Prof. Mag. Josef Lintz}
% TODO: Das "`Thema"' einsetzen
\Thema{Schriftliche Arbeiten im Vorwissenschaftlichen Zeitalter}
% TODO: Optional kann man einen Untertitel anführen
\Untertitel{Papyri und Pergamentrollen}

% TODO: erst bei der letzten Version das Abgabedatum anführen
% \Abgabedatum{}

\begin{document}

% Am Anfang keine Seitennummern
\frontmatter

% PDF-Lesezeichen für die Titelseite
\pdfbookmark[0]{Titelseite}{titlepage}
% Titelseite
\maketitle

\include{abstract}

\include{vorwort}

% Das Inhaltsverzeichnis soll ein PDF-Lesezeichen aber keinen Eintrag im
% Inhaltsverzeichnis haben.
\cleardoublepage\pdfbookmark[0]{\contentsname}{toc}
% Inhaltsverzeichnis
\tableofcontents

% Hier geht es los.
\mainmatter
\chapter{Mein allererstes Kapitel}
\section{Mein allererstes Unterkapitel}
\subsection{Ein Unterunterkapitel}

Hier ist ein Bisschen    
 {\Huge Text}. "`Hier direkte Rede."'
 
Neuer Absatz. \textbf{Hervorgehobener Text}. \textsf{Grotesk-Schrift}
\textsc{Kapitälchen}, \textit{Kursivschrift}.

Formeln lassen sich leicht setzen. $y=x^2$ entweder in der Zeile oder als
eigene Display-Umgebung.
\begin{equation}
f(x)=x^2
\end{equation}

\subsection{Zitierbeispiele}
Neulich habe ich mit Ex Perte darüber gesprochen. Er hat meine Ansichten
bestätigt.\zit[\pno\addabbrvspace\pageref{interessanteStelleImInterview}]{Perte}{Deswegen
soll dieser aber nicht heimreisen, sondern abermals die Einigkeit des
Urwalds in die aufgeregte Höhensonne schlagen.}

\zit{Scherz}{Wenn gleich die Nas, ob spitz, ob platt\lf zwei Flügel
(Nasenflügel) hat,\lf so hält sie doch nicht viel vom Fliegen,\lf das Laufen
scheint ihr mehr zu liegen.}

\section{Einheiten}
Heute hatte es nur \SI{7}{\celsius}. Der Bisamberg ist ca.\@ \SI{150}{\meter}
hoch. Die Lichtgeschwindigkeit ist ca.\@ \SI{3.0e8}{\meter\per\second}.
\vgls[31]{Lessing}[42]{Scherz}


\section{Text}
\begin{figure}\centering
\includegraphics[keepaspectratio,width=\textwidth,height=.25\textheight]{logo}
\caption{Das Logo}\label{logo}
\end{figure}
\newcommand{\unsinn}{Hier ist ein Absatz voll sinnlosem Text. Bitte erst nach
diesem Absatz weiterlesen. Hier kommt nichts mehr. Es folgen unterschiedlich lange Wörter.
Die Abruchbirne kringelte ihre Hürde in eine unbekannte Überschwänglichkeit, um
so die Sitzordnung der Fensterscheiben in der unteren Waldkante zu verjubeln.
Niemandem ist absichtlich zu kürzen, wessen Woligkeit hier in abermaligem
Abgesang aufgeschlagen ist. Deswegen soll dieser aber nicht heimreisen, sondern
abermals die Einigkeit des Urwalds in die aufgeregte Höhensonne schlagen.
Wenn nicht der hiesige Erdball des aberwitzigen Ungemachs aufgedrungene
Kröte wäre, entschließe ich mich zu unsachgemäßem Handlungsablauf.
Wiegleich zudem ein weiterer Honigkuchen ausbricht.\par}
\unsinn
Siehe auch \vref{logo}. \unsinn
\unsinn
\unsinn
\unsinn
\unsinn
\unsinn
\unsinn
\unsinn
\unsinn
\chapter{Hier gehts los}

Hier iat ein bisschen Text. 
Anführungszeichen macht eclipse automatisch sie gehen so: "`Text "`


Neuer Absatz. Absätze werden durch eine Leerzeile getrennt. Noch mehr Text. Mitti chillt sein Leben 
mit der Ukulele 

\zit[seite im Buch]{Dynamit}{Das Zitat}
\chapter{Mein drittes Kapitel}

Hallo ich bin gaaay


% Das Literaturverzeichnis
\input{literatur}

\listoffigures
% TODO: Wer keine Tabellen hat, muss das Tabellenverzeichnis entfernen!
% Bei kurzen Tabellenverzeichnissen kann man vielleicht
% Abbildungsverzeichnis und Tabellenverzeichnis auf einer Seite platzieren.
% \withoutclearpage unterdrückt die neue Seite.
\withoutclearpage{\listoftables}

% Gegebenenfalls ein Anhang
\appendix
\include{interview1}

\backmatter

%\pdfbookmark[0]{Erklärungen}{erkl}
\addchap{Erklärungen}
\section*{Selbstständigkeitserklärung}
\thispagestyle{plain}
Ich erkläre, dass ich diese vorwissenschaftliche Arbeit eigenständig
angefertigt und nur die im Literaturverzeichnis angeführten Quellen und
Hilfsmittel benutzt habe.

\vspace{2cm}\noindent Wien, \today

% TODO: Erkläre dich selbstständig selbstständig!
\vspace{2cm}\noindent\makeatletter\@AutorIn\makeatother

\vspace{2cm}\noindent

\section*{Informatikschwerpunkt}

Die vorliegende Arbeit erfüllt die Kriterien zur Abbildung des
Informatikschwerpunktes an der De La Salle Schule Strebersdorf, AHS.

\textbf{Begründung:} Die Arbeit wurde in \LaTeX{} mit entscheidenden
Kenntnissen zum Quelltext verfasst.\vspace{.5\baselineskip}

\noindent\textit{Geprüft am \ldots durch Mag. Rainer Zufall und Mag.
Ernst Haft}

\end{document}
