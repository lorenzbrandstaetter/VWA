% TODO: Soll das Vorwort im Inhaltsverzeichnis genannt werden?
% Mit \chapter*{Vorwort} wird eine Nennung im Inhaltsverzeichnis verhindert.
% \addchap{Vorwort}
\pdfbookmark[0]{Vorwort}{vorwort}
\chapter*{Vorwort}
Das Vorwort ist optional: d.\,h.\@ man muss kein Vorwort schreiben! Wer will,
kann das in dieser Form tun. Am Ende sollten Ort, Datum und der Name des Autors
des Vorworts angegeben werden. \vgl{Vorwort}

Hier ist ein Absatz voll sinnlosem Text. Bitte erst nach
diesem Absatz weiterlesen. Hier kommt nichts mehr. Es folgen unterschiedlich lange Wörter.
Die Abruchbirne kringelte ihre Hürde in eine unbekannte Überschwänglichkeit, um
so die Sitzordnung der Fensterscheiben in der unteren Waldkante zu verjubeln.
Niemandem ist absichtlich zu kürzen, wessen Woligkeit hier in abermaligem
Abgesang aufgeschlagen ist. Deswegen soll dieser aber nicht heimreisen, sondern
abermals die Einigkeit des Urwalds in die aufgeregte Höhensonne schlagen.
Wenn nicht der hiesige Erdball des aberwitzigen Ungemachs aufgedrungene
Kröte wäre, entschließe ich mich zu unsachgemäßem Handlungsablauf.
Wiegleich zudem ein weiterer Honigkuchen ausbricht.

\begin{flushleft}
Wien am \today
\end{flushleft}
\begin{flushright}
\makeatletter\@AutorIn\makeatother
\end{flushright}
